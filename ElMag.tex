% !TeX program = xelatex
% !TeX encoding = utf8
% !TeX root = ElMag.tex
% vim: sw=2 ts=2 et tw=80:

%% TODO: publish to CTAN
\documentclass[margin=small]{tex/hsrzf}

%%%%%%%%%%%%%%%%%%%%%%%%%%%%%%%%%%%%%%%%%%%%%%%%%%%
% Packages

%% TODO: publish to CTAN
\usepackage{tex/hsrstud}

%% Language configuration
\usepackage{polyglossia}
\setdefaultlanguage{english}

%% License configuration
\usepackage[
    type={CC},
    modifier={by-nc-sa},
    version={4.0},
    lang={english},
]{doclicense}

%% Theorems
\usepackage{amsthm}

%%%%%%%%%%%%%%%%%%%%%%%%%%%%%%%%%%%%%%%%%%%%%%%%%%%
% Metadata

\course{Electrical Engineering}
\module{ElMag}
\semester{Fall Semseter 2021}

\authoremail{naoki.pross@ost.ch}
\author{\textsl{Naoki Pross} -- \texttt{\theauthoremail}}

\title{\texttt{\themodule} Zusammenfassung}
\date{\thesemester}

%%%%%%%%%%%%%%%%%%%%%%%%%%%%%%%%%%%%%%%%%%%%%%%%%%%
% Macros and settings

%% Theorems
\newtheoremstyle{elmagzf} % name of the style to be used
  {\topsep}
  {\topsep}
  {}
  {0pt}
  {\bfseries}
  {.}
  { }
  { }

\theoremstyle{elmagzf}
\newtheorem{theorem}{Theorem}
\newtheorem{method}{Method}
\newtheorem{application}{Application}
\newtheorem{definition}{Definition}
\newtheorem{remark}{Remark}
\newtheorem{note}{Note}

%%%%%%%%%%%%%%%%%%%%%%%%%%%%%%%%%%%%%%%%%%%%%%%%%%%
% Document

\begin{document}

% use roman numberals for introductiory pages
\pagenumbering{roman}

\maketitle

% \begin{abstract}
% \end{abstract}

\tableofcontents

\section*{License}
\doclicenseThis

% actual content
\clearpage
\twocolumn
\setcounter{page}{1}
\pagenumbering{arabic}

\section{Vector Analysis Recap}

\subsection{Partial derivatives}

\begin{definition}[Partial derivative]
  A vector valued function \(f: \mathbb{R}^m\to\mathbb{R}\), with
  \(\vec{v}\in\mathbb{R}^m\), has a partial derivative with respect to \(v_i\)
  defined as
  \[
    \partial_{v_i} f(\vec{v})
      % = f_{v_i}(\vec{v})
      = \frac{\partial f}{\partial v_i}
      = \lim_{h\to 0} \frac{f(\vec{v} + h\uvec{e}_i) - f(\vec{v})}{h}
  \]
\end{definition}

\begin{theorem}[Integration of partial derivatives]
  Let \(f: \mathbb{R}^m\to\mathbb{R}\) be a partially differentiable function
  of many \(x_i\). When \(x_i\) is \emph{indipendent} with respect to all other
  \(x_j\) \((0 < j \leq m, j \neq i)\) then
  \[
    \int \partial_{x_i} f \,d x_i = f + C,
  \]
  where \(C\) is a function of \(x_1, \ldots, x_m\) but not of \(x_i\).
\end{theorem}

To illustrate the previous theorem, in a simpler case with \(f(x,y)\), we get
\[
  \int \partial_x f(x,y) \,dx = f(x, y) + C(y).
\]
Beware that this is valid only if \(x\) and \(y\) are indipendent.
If there is a relation \(x(y)\) or \(y(x)\) the above does not hold.

\subsection{Vector derivatives}

\begin{definition}[Gradient vector]
  The \emph{gradient} of a function \(f(\vec{x}), \vec{x}\in\mathbb{R}^m\) is a
  column vector containing the partial derivatives
  in each direction.
  \[
    \grad f (\vec{x}) = \sum_{i=1}^m \partial_{x_i} f(\vec{x}) \uvec{e}_i
      = \begin{pmatrix}
        \partial_{x_1} f(\vec{x}) \\
        \vdots \\
        \partial_{x_m} f(\vec{x}) \\
      \end{pmatrix}
  \]
\end{definition}

\begin{theorem}[Gradient in curvilinear coordinates]
  Let \(f: \mathbb{R}^3 \to \mathbb{R}\) be a scalar field. In cylindrical
  coordinates \((r,\phi,z)\)
  \[
    \grad f = \uvec{r}\,\partial_r f 
      + \uvec{\phi}\,\frac{1}{r}\partial_\phi f
      + \uvec{z}\,\partial_z f,
  \]
  and in spherical coordinates \((r,\theta,\phi)\)
  \[
    \grad f = \uvec{r}\,\partial_r f
      + \uvec{\theta}\,\frac{1}{r} \partial_\theta f
      + \uvec{\phi}\,\frac{1}{r \sin\theta} \partial_\phi f.
  \]
\end{theorem}

\begin{definition}[Divergence]
  Let \(\vec{F}: \mathbb{R}^m \to \mathbb{R}^m\) be a vector field.
  The divergence of \(\vec{F} = (F_{x_1},\ldots, F_{x_m})^t\) is
  \[
    \div\vec{F} = \sum_{i = 1}^m \partial_{x_i} F_{x_i} ,
  \]
  as suggested by the (ab)use of the dot product notation.
\end{definition}

\begin{theorem}[Divergence in curvilinear coordinates]
  Let \(\vec{F}: \mathbb{R}^3 \to \mathbb{R}^3\) be a field. In cylindrical
  coordinates \((r,\phi,z)\)
  \[
    \div \vec{F} = \frac{1}{r} \partial_r (r F_r)
      + \frac{1}{r}\partial_\phi F_\phi
      + \partial_z F_z,
  \]
  and in spherical coordinates \((r,\theta,\phi)\)
  \begin{align*}
    \div \vec{F} = \frac{1}{r^2} \partial_r (r^2 F_r)
      & + \frac{1}{r \sin\theta} \partial_\theta (\sin\theta F_\theta) \\
      & + \frac{1}{r \sin\theta} \partial_\phi F_\phi
  \end{align*}
\end{theorem}

\begin{theorem}[Divergence theorem, Gauss's theorem]
  Because the flux on the boundary \(\partial V\) of a volume \(V\) contains
  information of the field inside of \(V\), it is possible relate the two with
  \[
    \int_V \div \vec{F} \,dv = \oint_{\partial V} \vec{F} \dotp d\vec{s} .
  \]
\end{theorem}

\begin{definition}[Curl]
  Let \(\vec{F}\) be a vector field. In 2 dimensions
  \[
    \curl \vec{F} = \left(\partial_x F_y - \partial_y F_x\right)\uvec{z}.
  \]
  And in 3D
  \[
    \curl \vec{F} = \begin{pmatrix}
      \partial_y F_z - \partial_z F_y \\
      \partial_z F_x - \partial_x F_z \\
      \partial_x F_y - \partial_y F_x
    \end{pmatrix}
    = \begin{vmatrix}
      \uvec{x} & \uvec{y} & \uvec{z} \\
      \partial_x & \partial_y & \partial_z \\
      F_x & F_y & F_z
    \end{vmatrix} .
  \]
\end{definition}

\begin{definition}[Curl in curvilinear coordinates]
  Let \(\vec{F}: \mathbb{R}^3 \to \mathbb{R}^3\) be a field. In cylindrical
  coordinates \((r,\phi,z)\)
  \begin{align*}
    \curl \vec{F} =
      &\left(\frac{1}{r} \partial_\phi F_z - \partial_z F_\phi \right) \uvec{r} \\
      &+ \left(\partial_z F_r - \partial_r F_z \right) \uvec{\phi} \\
      &+ \frac{1}{r} \bigg[
        \partial_r (rF_\phi) - \partial_\phi F_r
        \bigg] \uvec{z},
  \end{align*}
  and in spherical coordinates \((r,\theta,\phi)\)
  \begin{align*}
      \curl \vec{F} =
        &\frac{1}{r \sin\theta} \bigg[
          \partial_\theta (\sin\theta F_\phi) - \partial_\phi F_\theta
        \bigg] \uvec{r} \\
        &+ \frac{1}{r} \bigg[
          \frac{1}{\sin\theta} \partial_\phi F_r - \partial_r (r F_\phi)
        \bigg] \uvec{\theta} \\
        &+ \frac{1}{r} \bigg[
          \partial_r (r F_\theta) - \partial_\theta F_r
        \bigg] \uvec{\phi} .
  \end{align*}
\end{definition}

\begin{theorem}[Stokes' theorem]
  \[
    \int_S \curl \vec{F} \dotp d\vec{s}
    = \oint_{\partial S} \vec{F} \dotp d\vec{r}
  \]
\end{theorem}

\subsection{Second vector derivatives}

\begin{definition}[Laplacian operator]
  A second vector derivative is so important that it has a special name.  For a
  scalar function \(f: \mathbb{R}^m \to \mathbb{R}\) the divergence of the
  gradient
  \[
    \laplacian f = \div (\grad f) = \sum_{i=1}^m \partial_{x_i}^2 f_{x_i}
  \]
  is called the \emph{Laplacian operator}.
\end{definition}

\begin{theorem}[Laplacian in curvilinear coordinates]
  Let \(f: \mathbb{R}^3 \to \mathbb{R}\) be a scalar field. In cylindrical
  coordinates \((r,\phi,z)\)
  \[
    \laplacian f = \frac{1}{r} \partial_r (r \partial_r f)
      + \frac{1}{r^2} \partial_\phi^2 f
      + \partial_z^2 f
  \]
  and in spherical coordinates \((r,\theta,\phi)\)
  \begin{align*}
    \laplacian f = 
      \frac{1}{r^2} \partial_r ( r^2 \partial_r f)
      & + \frac{1}{r^2\sin\theta} \partial_\theta (\sin\theta \partial_\theta f) \\
      & + \frac{1}{r^2 \sin^2 \theta} \partial_\phi^2 f.
  \end{align*}
\end{theorem}

\begin{definition}[Vector Laplacian]
  The Laplacian operator can be extended on a vector field \(\vec{F}\) to the 
  \emph{Laplacian vector} by applying the Laplacian to each component:
  \[
    \vlaplacian \vec{F} 
      = (\laplacian F_x)\uvec{x} 
      + (\laplacian F_y)\uvec{y} 
      + (\laplacian F_z)\uvec{z} .
  \]
  The vector Laplacian can also be defined as
  \[
    \vlaplacian \vec{F} = \grad (\div \vec{F}) - \curl (\curl \vec{F}).
  \]
\end{definition}

\begin{theorem}[Product rules and second derivatives]
  Let \(f,g\) be sufficiently differentiable scalar functions \(D
  \subseteq\mathbb{R}^m \to \mathbb{R}\) and \(\vec{A}, \vec{B}\) be
  sufficiently differentiable vector fields in \(\mathbb{R}^m\) (with \(m = 2\)
  or 3 for equations with the curl).
  \begin{itemize}
    \item Rules with the gradient
      \begin{align*}
        \grad (\div \vec{A}) &= \curl \curl \vec{A} + \vlaplacian \vec{A} \\
        \grad (f\cdot g) &= (\grad f)\cdot g + f\cdot \grad g \\
        \grad (\vec{A} \dotp \vec{B}) &= 
          (\vec{A} \dotp \grad) \vec{B}
          + (\vec{B} \dotp \grad) \vec{A} \\
          & + \vec{A} \crossp (\curl \vec{B})
          + \vec{B} \crossp (\curl \vec{A})
      \end{align*}
    \item Rules with the divergence
      \begin{align*}
        \div (\grad f) &= \laplacian f \\
        \div (\curl \vec{A}) &= 0 \\
        \div (f\cdot \vec{A}) &= (\grad f) \dotp \vec{A} + f\cdot (\div \vec{A}) \\
        \div (\vec{A}\crossp\vec{B}) &= (\curl \vec{A})\dotp \vec{B} 
          - \vec{A} \cdot (\curl\vec{B})
      \end{align*}
    \item Rules with the curl
      \begin{align*}
        \curl (\grad f) &= \vec{0} \\
        \curl (\curl \vec{A}) &= \grad (\div \vec{A}) - \vlaplacian \vec{A} \\
        \curl (\vlaplacian \vec{A}) &= \vlaplacian (\curl \vec{A}) \\
        \curl (f\cdot \vec{A}) &= (\grad f)\crossp \vec{A} + f\cdot \curl \vec{A} \\
        \curl (\vec{A}\crossp\vec{B}) &= 
          (\vec{B} \dotp \grad) \vec{A} - (\vec{A} \dotp \grad) \vec{B} \\
          &+ \vec{A} \dotp (\div \vec{B}) - \vec{B} \dotp (\div \vec{A})
      \end{align*}
  \end{itemize}
\end{theorem}

\section{Electrodynamics Recap}

\subsection{Maxwell's equations}

Maxwell's equations in matter in their integral form are
\begin{subequations}
  \begin{align}
    \oint_{\partial S} \vec{E} \dotp d\vec{l} &= -\frac{d}{dt} \int_S \vec{B} \dotp d\vec{s}, \\
    \oint_{\partial S} \vec{H} \dotp d\vec{l} &= \int_S \left(
      \vec{J} + \partial_t \vec{D}
    \right) \dotp d\vec{s}, \\
    \oint_{\partial V} \vec{D} \dotp d\vec{s} &= \int_V \rho \,dv, \\
    \oint_{\partial V} \vec{B} \dotp d\vec{s} &= 0.
  \end{align}
\end{subequations}
Where \(\vec{J}\) and \(\rho\) are the \emph{free current density} and
\emph{free charge density} respectively.

\subsection{Linear materials and boundary conditions}

Inside of so called isotropic linear materials fluxes and current
densities are proportional and parallel to the fields, i.e.
\begin{align*}
  \vec{D} &= \varepsilon \vec{E}, & \vec{J} &= \sigma \vec{E}, & \vec{B} &= \mu \vec{H}.
\end{align*}

Where two materials meet the following boundary conditions must be satisfied.
For the perpendicular component:
\begin{subequations}
  \begin{align}
    &\uvec{n} \dotp \vec{D}_1   = \uvec{n} \dotp \vec{D}_2 + \rho_s \\
    &\uvec{n} \dotp \vec{J}_1   = \uvec{n} \dotp \vec{J}_2 - \partial_t \rho_s \\
    &\uvec{n} \dotp \vec{B}_1   = \uvec{n} \dotp \vec{B}_2 - \partial_t \rho_s
  \end{align}
\end{subequations}
and for the tangential component:
\begin{subequations}
  \begin{align}
    &\uvec{n} \crossp \vec{E}_1 = \uvec{n} \crossp \vec{E}_2 \\
    &\uvec{n} \crossp \vec{H}_1 = \uvec{n} \crossp \vec{H}_2 + \vec{J}_s \\
    &\uvec{n} \crossp \vec{M}_1 = \uvec{n} \crossp \vec{M}_2 + \vec{J}_{s,m}
  \end{align}
\end{subequations}

\subsection{Potentials}

Because \(\vec{E}\) is often conservative (\(\curl \vec{E} = \vec{0}\)), and
\(\div \vec{B}\) is always zero, it is often useful to use \emph{potentials} to
describe these quantities instead. The electric scalar potential and magnetic
vector potentials are in their integral form:
\begin{align*}
  \varphi &= \int_\mathsf{A}^\mathsf{B} \vec{E} \dotp d\vec{l}, &
  \vec{A} &= \frac{\mu_0}{4\pi} \int_V \frac{\vec{J} dv}{R}.
\end{align*}
With differential operators:
\begin{align*}
  \vec{E} &= - \grad \varphi, &
  \mu_0 \vec{J} &= - \vlaplacian \vec{A}.
\end{align*}
By taking the divergence on both sides of the equation with the electric field
we get \(\rho/\varepsilon = - \laplacian \varphi\), which also contains the
Laplacian operator. We will study equations with of form in \S
\ref{sec:poisson}.  Potentials, are continuous quantities from any direction, so
we have the boundary conditions
\begin{align*}
  \varphi_1 &= \varphi_2, &
  \vec{A}_1 &= \vec{A}_2.
\end{align*}
However their derivatives inherit the discontinuity
\begin{align*}
  \uvec{n} \dotp \grad \varphi_1 &= \uvec{n} \dotp \grad \varphi_2, &
  \uvec{n} \dotp \grad \vec{A}_1 &= \uvec{n} \dotp \grad \vec{A}_2.
\end{align*}

% \subsection{Energy density}

\section{Boundary value problems}

\subsection{Steady-state flow analysis}
The equation for the steady-state analysis is
\begin{equation}
  \laplacian \varphi = 0 \quad\text{for}\quad \vec{r} \in \Omega,
\end{equation}
with its boundary conditions:
\(
  \varphi = 0 \text{ for } \vec{r}\in\Gamma_e,
  \varphi = U \text{ for } \vec{r}\in\Gamma_b,
  \nabla_\uvec{n} \varphi = 0 \text{ for } \vec{r}\in\Gamma_s.
\)

\subsection{Magnetostatic analysis}
The equation for the magnetostatic analysis is
\begin{equation}
  \vlaplacian \vec{A} = -\mu_0 \vec{J} \quad\text{for}\quad \vec{r} \in \Omega.
\end{equation}

\subsection{Magnetoquasistatic analysis}
The equation for the magnetoquasistatic analysis is
\begin{equation}
  \vlaplacian \vec{A} - \mu_0 \sigma \partial_t \vec{A}
    = -\mu_0 \vec{J}_q \quad\text{for}\quad \vec{r} \in \Omega,
\end{equation}
with its boundary conditions \(\uvec{n}\crossp\vec{A} = \vec{0}\) for \(\vec{r}\in\Gamma_i\).

\subsection{Electrodynamic analysis}
The equations for the electrodynamic analysis are
\begin{subequations}
  \begin{gather}
    \vlaplacian \vec{E}
      - \mu \sigma \partial_t \vec{E}
      - \mu \varepsilon \partial^2_t \vec{E} = \vec{0},
      \label{eqn:wave-equation-e} \\
    \vlaplacian \vec{H}
      - \mu \sigma \partial_t \vec{H}
      - \mu \varepsilon \partial^2_t \vec{H} = \vec{0}.
      \label{eqn:wave-equation-h}
  \end{gather}
\end{subequations}

\section{Laplace and Poisson's equations} \label{sec:poisson}

The so called \emph{Poisson's equation} has the form
\[
  \laplacian \varphi = - \frac{\rho}{\epsilon}.
\]
When the right side of the equation is zero, it is also known as \emph{Laplace's
equation}.

\subsection{Easy solutions of Laplace and Poisson's equations}

\paragraph{Geometry with zenithal and azimuthal symmetries (\"Ubung 2)}

Suppose we have a geometry where, using spherical coordinates, there is a
symmetry such that the solution does not depend on \(\phi\) or \(\theta\).
Then Laplace's equation reduces down to
\[
  \laplacian \varphi = \frac{1}{r^2} \partial_r ( r^2 \partial_r \varphi) = 0,
\]
which has solutions of the form
\[
  \varphi(r) = \frac{C_1}{r} + C_2.
\]

\paragraph{Geometry with azimuthal and translational symmetry (\"Ubung 3)}

Suppose that when using cylindrical coordinates, the solution does not depend
on \(\phi\) or \(z\). Then Laplace's equation becomes
\[
  \laplacian A_z = \frac{1}{r} \partial_r (r \partial_r A_z) = 0.
\]

\section{Solved problems}

\paragraph{Concentric rings}
\begin{quote}
  Two concentric rings of conductor (\(\sigma \to \infty\)) have a tall
  rectangular cross section (ignore boundary effects). The distance from the
  center to the outer edge of the first ring is \(a\), and the distance to the
  inner edge of the second is \(b > a\).  In other words, the air gap between
  the two rings has width \(b - a\).  Both rings are coated with a dielectric
  (\(\varepsilon_{r1}, \varepsilon_{r3} > 1\)) of thickness \(d\). The inner
  ring is grounded, while the second has a voltage \(u > 0\).  Compute the
  electric field \(\vec{E}\) between the rings.
\end{quote}

To solve this problem we let \(C\) be a cylinder centered on the axis of the
rings, with radius \(r\) and height \(h\). Because the outer ring has a higher
potential (\(u > 0\)) we can assume that \(\uvec{E} = -\uvec{r}\). The we use
Gauss's law:
\[
  Q = \oint_C \vec{D} \dotp d\vec{s} 
    = \oint_C \varepsilon(r) \vec{E} \dotp d\vec{s}
    = - 2 \pi r h \varepsilon(r) E,
\]
where
\[
  \varepsilon(r) = \begin{cases}
    \varepsilon_{r1}\varepsilon_0  & a \leq r < a + d     \\
    \varepsilon_0                  & a + d \leq r < b - d \\
    \varepsilon_{r3}\varepsilon_0  & b - d \leq r < b     \\
  \end{cases}
\]
thus
\[
  \vec{E} = \frac{-\uvec{z} Q}{2 \pi h r\varepsilon(r)}.
\]
The unknown charge \(Q\) can be expressed in terms of the voltage \(u\) since
\[
  u = \int_b^a \vec{E} \dotp d\vec{r}
   = \frac{Q}{2\pi h} \int_a^b \frac{dr}{r\varepsilon(r)}.
\]
Therefore
\[
  Q = 2\pi h u \left( \int_a^b \frac{dr}{r\varepsilon(r)} \right)^{-1},
\]
which when substituted into \(\vec{E}\) gives the solution
\[
  \vec{E} = \frac{-u \uvec{z}}{r\varepsilon(r)} \left(
    \int_a^b \frac{dr}{r\varepsilon(r)}
  \right)^{-1}.
\]

We shall now solve again for \(\vec{E} = -\grad \varphi\), but this time using a
boundary value problem in terms of \(\varphi\). By taking the divergence on both
sides we get
\[
  \div \vec{E} = - \div \grad \varphi \implies
  \frac{\rho}{\varepsilon} = - \laplacian \varphi,
\]
i.e. Poisson's equation. Since between \(a\) and \(b\) the charge density is
zero and thanks to cylindrical symmetry the problem reduces down to:
\begin{align}
  0 = - \laplacian \varphi
    &= - \frac{1}{r} \partial_r ( r \partial_r \varphi ) \nonumber \\
  \implies 0 &= \frac{d}{dr} \left( r \frac{d\varphi}{dr} \right).
  \label{eqn:classic-phi-ode}
\end{align}
Before solving \eqref{eqn:classic-phi-ode} we need to define the boundary
conditions.  First, the voltage is fixed at the electrodes to zero and \(u\).
Secondly, physically the electric potential must be continuous. Third, the
derivative must also be continuous because the flux \(\vec{D} = - \varepsilon
\grad \varphi\) is continuous.  Thus between \(a\) and \(b\) there are 3
regions: \(\varphi_1\) in the first dielectic, \(\varphi_2\) in the air between
and \(\varphi_3\) in the second dielectric.  Algebraically:
\begin{subequations}
  \begin{align}
    \varphi_1(a) &= 0, \label{eqn:classic-phi-bound-1z} \\
    \varphi_3(b) &= u, \label{eqn:classic-phi-bound-3u} \\
    \varphi_1(a + d) &= \varphi_2(a + d), \\
    \varphi_2(b - d) &= \varphi_3(b - d), \\
    \varepsilon_{r1} \partial_r \varphi_1(a + d) &= \partial_r \varphi_2(a + d),
      \label{eqn:classic-phi-bound-d12} \\
    \partial_r \varphi_2(b - d) &= \varepsilon_{r3} \partial_r \varphi_3(b - d).
      \label{eqn:classic-phi-bound-d23}
  \end{align}
\end{subequations}
The ODE \eqref{eqn:classic-phi-ode} has solutions of the form:
\begin{equation}
  \varphi_i = K_i + C_i \ln r.
\end{equation}
The unknowns can be found by using the boundary conditions.

\paragraph{Grounded pole}
\begin{quote}
  A pole (mast) is anchored to the ground using a metal half sphere (\(\sigma
  \to \infty\)) of radius \(a\), that is surrounded by the ground which has a
  conductivity \(\sigma(r)\). What is the resistance of the ground?
\end{quote}

We are looking for \(R = U / I\), so we try to find an expression \(U(I)/I\). To
solve this problem we consider a current \(I\) on the pole towards the ground,
which creates a current density \(\vec{J}\) on a half sphere \(S\) of
radius \(r\), that is related to \(I\) by:
\begin{align*}
  I = \int_{S} \vec{J} \dotp d\vec{s}
  &= \int_{S} \sigma(r) \vec{E} \dotp d\vec{s} \\
  &= \sigma(r) E \int_{S} \, r^2 \sin(\theta) d\theta d\phi.
\end{align*}
Because \(\vec{E}\) and \(\sigma(r)\) are constant on the surface (do not depend
on \(\theta\) or \(\phi\)), we can solve for the electric field to get
\[
  \vec{E}(r) = \frac{I \uvec{r}}{2 \pi r^2 \sigma(r)}.
\]
With the electric field as a function of \(I\), we can now find the voltage as a
function of the current:
\[
  U(I) = \int_a^\infty \vec{E}(r) \dotp d\vec{r}
       = \frac{I}{2\pi} \int_a^\infty \frac{dr}{r^2 \sigma(r)},
\]
and finally divide by \(I\) to get
\begin{equation}
  R = \frac{1}{2\pi} \int_a^\infty \frac{dr}{r^2 \sigma(r)}.
\end{equation}

Reformulating this problem as a boundary value problem, we look for the
potential distribution \(\varphi\) to find \(R = U / I(U)\). This problem is
formulated with the Laplace equation
\[
  \laplacian \varphi = \frac{1}{r^2}\partial_r (r^2 \partial_r \varphi) = 0,
\]
and the boundary conditions:
\begin{subequations}
  \begin{align}
    \varphi(a) &= U, \\
    \varphi(r \to \infty) &= 0,
  \end{align}
  and the requirements for the derivative \(- \sigma(r) \grad \varphi = \vec{J}\)
  to be continuous. If we let
  \[
    \sigma(r) = \begin{cases}
      \sigma_1, & a \leq r < b \\
      \sigma_2, & b \leq r \\
    \end{cases}
  \]
  then we get the boundary condition:
  \begin{equation}
    \sigma_1 \partial_r \varphi_1 = \sigma_2 \partial_r \varphi_2,
  \end{equation}
  where \(\varphi_1\) is in the first conductor, where the potential at its end
  equals \(U\), and \(\varphi_2\) is in the second conductor.
\end{subequations}

\paragraph{Magnetic field around a conductor}
\begin{quote}
  Though a conductor of radius \(a\) flows a homogeneous currrent density
  \(J_0\uvec{z}\). Compute the flux density \(\vec{B}\).
\end{quote}

We let \(C\) be an Amperian contour, that is a disc of radius \(r\) around the
axis of the conductor, and then simply use Ampere's law 
\[
  \oint_{\partial C} \vec{H} \dotp d\vec{l} = \int_C \vec{J}(r) \dotp d\vec{s}.
\]
Because of symmetry, the left side simplifies down to \(2\pi r H\), thus
\[
  \vec{H} = \frac{\uvec{\phi}}{2\pi r} \int_C \vec{J}(r) \dotp d\vec{s} =
    \begin{cases}
      r J_0 \uvec{\phi}, & r \leq a \\
      a^2 J_0 \uvec{\phi} / (2r) , & r > a \\
    \end{cases}
\]
and finally \(\vec{B} = \mu \vec{H}\). So the flux density is linear in the
material, and proportional to \(1/r\) outside.

If we now wish to formulate this as a boundary value problem, we apply Stoke's
theorem to Ampere's law to get
\[
  \int_S \curl \vec{H} \dotp d\vec{s} = \int_S \vec{J} \dotp d\vec{s},
\]
then by removing the integrals and multiplying both sides by \(\mu\) we get
\begin{align*}
  \mu \vec{J} &= \curl \mu \vec{H} = \curl \curl \vec{A} \\
    &= \big(
        \underbrace{\grad ( \div \vec{A} )}_{= 0}
        - \vlaplacian \vec{A}
      \big) = -\vlaplacian \vec{A}.
\end{align*}
We know that in \(\uvec{A} = \uvec{J} = \uvec{z}\), so in cylindrical
coordinates the vector Laplacian simplifies down to a single dimension
\begin{equation} \label{eqn:classic-az-ode}
  \mu J(r) = - \frac{1}{r} \partial_r ( r \partial_r A_z ),
\end{equation}
an ODE in \(r\).  The boundary conditions for this problem are: that at the
border between the conductor the \(A_z\) is continuous, and the derivative
\(\partial_r A_z\) is also continuous (because \(\vec{B} = \curl \vec{A}\) is
continuous), lastly when \(r = 0\) we want \(A_z\) to not be infinite, so we set
it to zero.  Algebraically, if 1 is inside the conductor and 2 outside:
\begin{subequations}
  \begin{align}
    A_{z1} (0) &= 0 , \label{eqn:classic-az-finite} \\
    A_{z1} (a) &= A_{z2} (a) , \label{eqn:classic-az-cont} \\
    \partial_r A_{z1} (a) &= \partial_r A_{z2} (a) , \label{eqn:classic-az-bcont}
  \end{align}
\end{subequations}
We shall now solve \eqref{eqn:classic-az-ode}. Because \(J(r) = J_0\) for \(r \leq
a\) and zero elsewhere this is easy. First we do the outside
\[
  0 = -\frac{1}{r} \frac{d}{dr} \left( r \frac{dA_{z2}}{dr} \right)
  \implies A_{z2} = K_2 + C_2 \ln r,
\]
then similarly the inside gives
\begin{align*}
  \mu J_0 = & -\frac{1}{r} \frac{d}{dr} \left( r \frac{dA_{z1}}{dr} \right) \\
  &\implies A_{z1} = K_1 - \frac{\mu J_0 r^2}{4} + C_1 \ln r .
\end{align*}
Finally with the boundary conditions we find the values for \(C_1, C_2, K_1,
K_2\):
\begin{itemize}
  \item Because of \eqref{eqn:classic-az-finite} \(C_1 = K_1 = 0\);
  \item With \eqref{eqn:classic-az-cont} and \eqref{eqn:classic-az-bcont} we get
    a system of equations
    \[
      -\frac{\mu J_0 a^2}{4} = K_2 + C_2 \ln a, \qquad
      \frac{\mu J_0 a}{2} = \frac{C_2}{a}.
    \]
    With the latter we find that \(C_2 = \mu J_0 a^2 / 2\), and thus
    \(K_2 = -\mu J_0 a^2 (1 - 2\ln a) / 2\)
\end{itemize}
The final result is 
\begin{align*}
  \vec{A} = \begin{cases}
    -\mu J_0 r^2 \, \uvec{z} / 2 ,                   & r \leq a \\
    -\mu J_0 a^2 (1 + 2 \ln (r/a)) \, \uvec{z} / 4 , & r > a.   \\
  \end{cases}
\end{align*}
To get the flux density we use \(\vec{B} = \curl \vec{A}\):
\[
  \vec{B} = \uvec{\phi} \partial_r A_z = \begin{cases}
    -\mu J_0 r \uvec{\phi}, & r \leq a \\
    -\mu J_0 a^2 \uvec{\phi} / (2r) & r > a .\\
  \end{cases}
\]

% \paragraph{Infinite plane}

\paragraph{Planar electromagnetic wave}

\newcommand{\vwave}[1]{\vec{\tilde{#1}}}

\begin{quote}
  An electromagnetic wave \(\vwave{E} = \tilde{E}(x) \uvec{y}\) in air
  (\(\sigma = 0, \varepsilon_r = 1\)) hits a dielectric wall (\(\sigma \approx
  0, \varepsilon_r > 1\)) on the \(zy\) plane at \(x = 0\). Compute the
  reflected and transmitted waves \(\vwave{E}_r\), \(\vwave{H}_r\)
  \(\vwave{E}_t\) and \(\vwave{H}_t\).
\end{quote}

This problem is solved using the wave equation \eqref{eqn:wave-equation-e} in the
frequency domain:
\[
  \vlaplacian \vwave{E} + \omega^2 \mu \varepsilon \vwave{E} = 0.
\]
Because \(\vwave{E} = \tilde{E}(x) \uvec{y}\) the problem is reduced to an ODE:
\begin{equation}
  \frac{d^2 \tilde{E}}{dx^2} + \omega^2 \mu \varepsilon \tilde{E} = 0.
\end{equation}
With the ansatz \(\tilde{E} = Ce^{\lambda x}\), we get the characteristic
polynomial and its solutions:
\[
  (\lambda^2 + \omega^2\mu\varepsilon) = 0
    \implies \lambda_{1,2} = \pm j\omega\sqrt{\mu\varepsilon}.
\]
Thus the general solution is
\[
  \tilde{E} = C_t e^{-j\omega x\sqrt{\mu\varepsilon}}
    + C_r e^{j\omega x\sqrt{\mu\varepsilon}}
    = \tilde{E}_t + \tilde{E}_r.
\]
By inspecting this result we see that there are two waves: one traveling in the
positive \(x\) direction (i.e. \(\exp(-j\omega x\sqrt{\mu\varepsilon})\)), and
the other traveling in the negative \(x\) direction (\(\exp(+j\omega
x\sqrt{\mu\varepsilon})\)). Further analyzing the exponent, we see that
dimensionally we have distance over time times the units of
\(\sqrt{\mu\varepsilon}\). Because the exponent hast to be dimensionless, we can
infer that the rest must be the reciprocal of a velocity:
\[
  \omega x \sqrt{\mu\varepsilon} = \frac{\omega x}{v}, 
  \text{ where } v = \frac{1}{\sqrt{\mu\varepsilon}}
  = \frac{c_0}{\sqrt{\mu_r\varepsilon_r}}.
\]
Thus in the two materials (air and dielectric) we have the equations:
\begin{subequations}
  \begin{align}
    \tilde{E}_1 &= C_{t1} e^{-j\omega x / c_0} + C_{r1} e^{j\omega x / c_0}, \\
    \tilde{E}_2 &= C_{t2} e^{-j\omega x \sqrt{\varepsilon_r} / c_0}.
  \end{align}
\end{subequations}
Since in the dielectric we cannot possibly have a wave in the negative \(x\)
direction, we can assume that \(C_{r2} = 0\).  The boundary conditions are given
by the physical conditions \(\uvec{x} \crossp \vwave{E}_1 = \uvec{x} \crossp
\vwave{E}_2\) and \(\uvec{x} \crossp \vwave{H}_1 = \uvec{x} \crossp
\vwave{H}_2\), thus since
\[
  \vwave{H} = \frac{-1}{j\omega \mu_0} \curl \vwave{E}
  = \frac{- \uvec{z}}{j\omega \mu_0} \frac{d\tilde{E}}{dx}
\]
we get the boundary conditions
\begin{subequations}
\end{subequations}

\section{Finite element method}

The finite element method (FEM) is a popular numerical method for solving
partial differential equations, i.e. a boundary value problem. FEM works by
discretizing space into a so called \emph{mesh}, and computing the physical
quantities (fields) only on it. This reduces a continuous problem (of
inifinitely many infinitely small changes) into a linear system of equations
that is solved using linear algebra.

\subsection{Electrostatic FEM}

For an electrostatic boundary value problem, we are trying to solve \(\laplacian
\varphi = 0\) in a domain \(\Omega \subset \mathbb{R}^n\), given a charge
density distribution \(\rho(\vec{r})\). From physics we know that in nature
(solution) the energy of the system will be minimzed. So we need the energy,
which is computed with
\[
  W_e = \int_\Omega \rho \, \varphi \, ds.
\]
Because \(\rho = \div \vec{D}\), we can rewrite it as
\begin{align*}
  W_e &= \int_\Omega (\div \vec{D}) \, \varphi \, ds
       = \int_\Omega \div (\vec{D} \varphi) - \vec{D} \dotp (\grad \varphi) \, ds \\
      &= \underbrace{\oint_{\partial \Omega} \vec{D} \varphi \dotp d\vec{s}}_{
        \varphi(\vec{r} \in \partial \Omega) = 0
      }
         + \int_\Omega \vec{D} \dotp (-\grad \varphi) \, ds \\
      &= \int_\Omega \vec{D} \dotp \vec{E} \, ds
       = \int_\Omega \varepsilon E^2 \, ds
       = \int_\Omega \varepsilon (\grad \varphi)^2 \, ds
\end{align*}
Now we can write the energy \(W_e\) both using charge density and potentials,
as well as with fields. We can combine both into a single expression by taking
half from each:
\begin{equation}
  W_e = \frac{1}{2} \int_\Omega \rho \, \varphi \, ds
        + \frac{1}{2} \int_\Omega \varepsilon (\grad \varphi)^2 \, ds.
\end{equation}
This energy must be minimized, so \(\grad W_e = \vec{0}\), but this cannot be
practically solved, it is necessary to approximate \(\varphi\). Of the domain
\(\Omega\) only a finite number of points \(M\) is taken, and the rest of omega
is interpolated using a function \(N(\vec{r})\)
\begin{equation}
  \varphi^e = \sum_k N_k \varphi^e
\end{equation}

\end{document}
