% !TeX program = xelatex
% !TeX encoding = utf8
% !TeX root = ElMag.tex
% vim: sw=2 ts=2 et:

%% TODO: publish to CTAN
\documentclass[margin=normal]{tex/hsrzf}

%%%%%%%%%%%%%%%%%%%%%%%%%%%%%%%%%%%%%%%%%%%%%%%%%%%
% Packages

%% TODO: publish to CTAN
\usepackage{tex/hsrstud}

%% Language configuration
\usepackage{polyglossia}
\setdefaultlanguage{english}

%% License configuration
\usepackage[
    type={CC},
    modifier={by-nc-sa},
    version={4.0},
    lang={english},
]{doclicense}

%% Theorems
\usepackage{amsthm}

%%%%%%%%%%%%%%%%%%%%%%%%%%%%%%%%%%%%%%%%%%%%%%%%%%%
% Metadata

\course{Electrical Engineering}
\module{ElMag}
\semester{Fall Semseter 2021}

\authoremail{naoki.pross@ost.ch}
\author{\textsl{Naoki Pross} -- \texttt{\theauthoremail}}

\title{\texttt{\themodule} Zusammenfassung}
\date{\thesemester}

%%%%%%%%%%%%%%%%%%%%%%%%%%%%%%%%%%%%%%%%%%%%%%%%%%%
% Macros and settings

%% Theorems
\newtheoremstyle{elmagzf} % name of the style to be used
  {\topsep}
  {\topsep}
  {}
  {0pt}
  {\bfseries}
  {.}
  { }
  { }

\theoremstyle{elmagzf}
\newtheorem{theorem}{Theorem}
\newtheorem{method}{Method}
\newtheorem{application}{Application}
\newtheorem{definition}{Definition}
\newtheorem{remark}{Remark}
\newtheorem{note}{Note}

%%%%%%%%%%%%%%%%%%%%%%%%%%%%%%%%%%%%%%%%%%%%%%%%%%%
% Document

\begin{document}

% use roman numberals for introductiory pages
\pagenumbering{roman}

\maketitle

% \begin{abstract}
% \end{abstract}

\tableofcontents

\section*{License}
\doclicenseThis

% actual content
\clearpage
\twocolumn
\setcounter{page}{1}
\pagenumbering{arabic}

\section{Vector Analysis Recap}

\subsection{Partial derivatives}

\begin{definition}[Partial derivative]
  A vector valued function \(f: \mathbb{R}^m\to\mathbb{R}\), with
  \(\vec{v}\in\mathbb{R}^m\), has a partial derivative with respect to \(v_i\)
  defined as
  \[
    \partial_{v_i} f(\vec{v})
      % = f_{v_i}(\vec{v})
      = \frac{\partial f}{\partial v_i}
      = \lim_{h\to 0} \frac{f(\vec{v} + h\vec{e}_i) - f(\vec{v})}{h}
  \]
\end{definition}

\begin{theorem}[Integration of partial derivatives]
  Let \(f: \mathbb{R}^m\to\mathbb{R}\) be a partially differentiable function
  of many \(x_i\). When \(x_i\) is \emph{indipendent} with respect to all other
  \(x_j\) \((0 < j \leq m, j \neq i)\) then
  \[
    \int \partial_{x_i} f \,d x_i = f + C,
  \]
  where \(C\) is a function of \(x_1, \ldots, x_m\) but not of \(x_i\).
\end{theorem}

To illustrate the previous theorem, in a simpler case with \(f(x,y)\), we get
\[
  \int \partial_x f(x,y) \,dx = f(x, y) + C(y).
\]
Beware that this is valid only if \(x\) and \(y\) are indipendent.
If there is a relation \(x(y)\) or \(y(x)\) the above does not hold.

\subsection{Vector derivatives}

\begin{definition}[Gradient vector]
  The \emph{gradient} of a function \(f(\vec{x}), \vec{x}\in\mathbb{R}^m\) is a
  column vector containing the partial derivatives
  in each direction.
  \[
    \grad f (\vec{x}) = \sum_{i=1}^m \partial_{x_i} f(\vec{x}) \vec{e}_i
      = \begin{pmatrix}
        \partial_{x_1} f(\vec{x}) \\
        \vdots \\
        \partial_{x_m} f(\vec{x}) \\
      \end{pmatrix}
  \]
\end{definition}

\begin{theorem}[Gradient in curvilinear coordinates]
  Let \(f: \mathbb{R}^3 \to \mathbb{R}\) be a scalar field. In cylindrical
  coordinates \((r,\phi,z)\)
  \[
    \grad f = \uvec{r}\,\partial_r f 
      + \uvec{\phi}\,\frac{1}{r}\partial_\phi f
      + \uvec{z}\,\partial_z f,
  \]
  and in spherical coordinates \((r,\theta,\phi)\)
  \[
    \grad f = \uvec{r}\,\partial_r f
      + \uvec{\theta}\,\frac{1}{r} \partial_\theta f
      + \uvec{\phi}\,\frac{1}{r \sin\theta} \partial_\phi f.
  \]
\end{theorem}

\begin{definition}[Divergence]
  Let \(\vec{F}: \mathbb{R}^m \to \mathbb{R}^m\) be a vector field.
  The divergence of \(\vec{F} = (F_{x_1},\ldots, F_{x_m})^t\) is
  \[
    \div\vec{F} = \sum_{i = 1}^m \partial_{x_i} F_{x_i} ,
  \]
  as suggested by the (ab)use of the dot product notation.
\end{definition}

\begin{theorem}[Divergence in curvilinear coordinates]
  Let \(\vec{F}: \mathbb{R}^3 \to \mathbb{R}^3\) be a field. In cylindrical
  coordinates \((r,\phi,z)\)
  \[
    \div \vec{F} = \frac{1}{r} \partial_r (r F_r)
      + \frac{1}{r}\partial_\phi F_\phi
      + \partial_z F_z,
  \]
  and in spherical coordinates \((r,\theta,\phi)\)
  \begin{align*}
    \div \vec{F} = \frac{1}{r^2} \partial_r (r^2 F_r)
      & + \frac{1}{r \sin\theta} \partial_\theta (\sin\theta F_\theta) \\
      & + \frac{1}{r \sin\theta} \partial_\phi F_\phi
  \end{align*}
\end{theorem}

\begin{theorem}[Divergence theorem, Gauss's theorem]
  Because the flux on the boundary \(\partial V\) of a volume \(V\) contains
  information of the field inside of \(V\), it is possible relate the two with
  \[
    \int_V \div \vec{F} \,dv = \oint_{\partial V} \vec{F} \dotp d\vec{s} .
  \]
\end{theorem}

\begin{definition}[Curl]
  Let \(\vec{F}\) be a vector field. In 2 dimensions
  \[
    \curl \vec{F} = \left(\partial_x F_y - \partial_y F_x\right)\uvec{z}.
  \]
  And in 3D
  \[
    \curl \vec{F} = \begin{pmatrix}
      \partial_y F_z - \partial_z F_y \\
      \partial_z F_x - \partial_x F_z \\
      \partial_x F_y - \partial_y F_x
    \end{pmatrix}
    = \begin{vmatrix}
      \uvec{x} & \uvec{y} & \uvec{z} \\
      \partial_x & \partial_y & \partial_z \\
      F_x & F_y & F_z
    \end{vmatrix} .
  \]
\end{definition}

\begin{definition}[Curl in curvilinear coordinates]
  Let \(\vec{F}: \mathbb{R}^3 \to \mathbb{R}^3\) be a field. In cylindrical
  coordinates \((r,\phi,z)\)
  \begin{align*}
    \curl \vec{F} =
      &\left(\frac{1}{r} \partial_\phi F_z - \partial_z F_\phi \right) \uvec{r} \\
      &+ \left(\partial_z F_r - \partial_r F_z \right) \uvec{\phi} \\
      &+ \frac{1}{r} \bigg[
        \partial_r (rF_\phi) - \partial_\phi F_r
        \bigg] \uvec{z},
  \end{align*}
  and in spherical coordinates \((r,\theta,\phi)\)
  \begin{align*}
      \curl \vec{F} =
        &\frac{1}{r \sin\theta} \bigg[
          \partial_\theta (\sin\theta F_\phi) - \partial_\phi F_\theta
        \bigg] \uvec{r} \\
        &+ \frac{1}{r} \bigg[
          \frac{1}{\sin\theta} \partial_\phi F_r - \partial_r (r F_\phi)
        \bigg] \uvec{\theta} \\
        &+ \frac{1}{r} \bigg[
          \partial_r (r F_\theta) - \partial_\theta F_r
        \bigg] \uvec{\phi} .
  \end{align*}
\end{definition}

\begin{theorem}[Stokes' theorem]
  \[
    \int_S \curl \vec{F} \dotp d\vec{s}
    = \oint_{\partial S} \vec{F} \dotp d\vec{r}
  \]
\end{theorem}

\subsection{Second vector derivatives}

\begin{definition}[Laplacian operator]
  A second vector derivative is so important that it has a special name.  For a
  scalar function \(f: \mathbb{R}^m \to \mathbb{R}\) the divergence of the
  gradient
  \[
    \laplacian f = \div (\grad f) = \sum_{i=1}^m \partial_{x_i}^2 f_{x_i}
  \]
  is called the \emph{Laplacian operator}.
\end{definition}

\begin{theorem}[Laplacian in curvilinear coordinates]
  Let \(f: \mathbb{R}^3 \to \mathbb{R}\) be a scalar field. In cylindrical
  coordinates \((r,\phi,z)\)
  \[
    \laplacian f = \frac{1}{r} \partial_r (r \partial_r f)
      + \frac{1}{r^2} \partial_\phi^2 f
      + \partial_z^2 f
  \]
  and in spherical coordinates \((r,\theta,\phi)\)
  \begin{align*}
    \laplacian f = 
      \frac{1}{r^2} \partial_r ( r^2 \partial_r f)
      & + \frac{1}{r^2\sin\theta} \partial_\theta (\sin\theta \partial_\theta f) \\
      & + \frac{1}{r^2 \sin^2 \theta} \partial_\phi^2 f.
  \end{align*}
\end{theorem}

\begin{definition}[Vector Laplacian]
  The Laplacian operator can be extended on a vector field \(\vec{F}\) to the 
  \emph{Laplacian vector} by applying the Laplacian to each component:
  \[
    \vlaplacian \vec{F} 
      = (\laplacian F_x)\uvec{x} 
      + (\laplacian F_y)\uvec{y} 
      + (\laplacian F_z)\uvec{z} .
  \]
  The vector Laplacian can also be defined as
  \[
    \vlaplacian \vec{F} = \grad (\div \vec{F}) - \curl (\curl \vec{F}).
  \]
\end{definition}

\begin{theorem}[Product rules and second derivatives]
  Let \(f,g\) be sufficiently differentiable scalar functions \(D
  \subseteq\mathbb{R}^m \to \mathbb{R}\) and \(\vec{A}, \vec{B}\) be
  sufficiently differentiable vector fields in \(\mathbb{R}^m\) (with \(m = 2\)
  or 3 for equations with the curl).
  \begin{itemize}
    \item Rules with the gradient
      \begin{align*}
        \grad (\div \vec{A}) &= \curl \curl \vec{A} + \vlaplacian \vec{A} \\
        \grad (f\cdot g) &= (\grad f)\cdot g + f\cdot \grad g \\
        \grad (\vec{A} \dotp \vec{B}) &= 
          (\vec{A} \dotp \grad) \vec{B}
          + (\vec{B} \dotp \grad) \vec{A} \\
          & + \vec{A} \crossp (\curl \vec{B})
          + \vec{B} \crossp (\curl \vec{A})
      \end{align*}
    \item Rules with the divergence
      \begin{align*}
        \div (\grad f) &= \laplacian f \\
        \div (\curl \vec{A}) &= 0 \\
        \div (f\cdot \vec{A}) &= (\grad f) \dotp \vec{A} + f\cdot (\div \vec{A}) \\
        \div (\vec{A}\crossp\vec{B}) &= (\curl \vec{A})\dotp \vec{B} 
          - \vec{A} \cdot (\curl\vec{B})
      \end{align*}
    \item Rules with the curl
      \begin{align*}
        \curl (\grad f) &= \vec{0} \\
        \curl (\curl \vec{A}) &= \grad (\div \vec{A}) - \vlaplacian \vec{A} \\
        \curl (\vlaplacian \vec{A}) &= \vlaplacian (\curl \vec{A}) \\
        \curl (f\cdot \vec{A}) &= (\grad f)\crossp \vec{A} + f\cdot \curl \vec{A} \\
        \curl (\vec{A}\crossp\vec{B}) &= 
          (\vec{B} \dotp \grad) \vec{A} - (\vec{A} \dotp \grad) \vec{B} \\
          &+ \vec{A} \dotp (\div \vec{B}) - \vec{B} \dotp (\div \vec{A})
      \end{align*}
  \end{itemize}
\end{theorem}

\section{Electrodynamics Recap}

\subsection{Maxwell's equations}

Maxwell's equations in matter in their integral form are
\begin{subequations}
  \begin{align}
    \oint_{\partial S} \vec{E} \dotp d\vec{l} &= -\frac{d}{dt} \int_S \vec{B} \dotp d\vec{s}, \\
    \oint_{\partial S} \vec{H} \dotp d\vec{l} &= \int_S \left(
      \vec{J} + \partial_t \vec{D}
    \right) \dotp d\vec{s}, \\
    \oint_{\partial V} \vec{D} \dotp d\vec{s} &= \int_V \rho \,dv, \\
    \oint_{\partial V} \vec{B} \dotp d\vec{s} &= 0.
  \end{align}
\end{subequations}
Where \(\vec{J}\) and \(\rho\) are the \emph{free current density} and
\emph{free charge density} respectively.

\subsection{Linear materials and boundary conditions}

Inside of so called isotropic linear materials fluxes and current
densities are proportional and parallel to the fields, i.e.
\begin{align*}
  \vec{D} &= \epsilon \vec{E}, & \vec{J} &= \sigma \vec{E}, & \vec{B} &= \mu \vec{H}.
\end{align*}

Where two materials meet the following boundary conditions must be satisfied:
\begin{align*}
  &\uvec{n} \dotp \vec{D}_1 = \uvec{n} \dotp \vec{D}_2 + \rho_s &
    &\uvec{n} \crossp \vec{E}_1 = \uvec{n} \crossp \vec{E}_2 \\
  &\uvec{n} \dotp \vec{J}_1 = \uvec{n} \dotp \vec{J}_2 - \partial_t \rho_s &
    &\uvec{n} \crossp \vec{H}_1 = \uvec{n} \crossp \vec{H}_2 + \vec{J}_s \\
  &\uvec{n} \dotp \vec{B}_1 = \uvec{n} \dotp \vec{B}_2 - \partial_t \rho_s &
    &\uvec{n} \crossp \vec{M}_1 = \uvec{n} \crossp \vec{M}_2 + \vec{J}_{s,m}
\end{align*}

\subsection{Potentials}

Because \(\vec{E}\) is often conservative (\(\curl \vec{E} = \vec{0}\)), and
\(\div \vec{B}\) is always zero, it is often useful to use \emph{potentials} to
describe these quantities instead. The electric scalar potential and magnetic
vector potentials are in their integral form:
\begin{align*}
  \varphi &= \int_\mathsf{A}^\mathsf{B} \vec{E} \dotp d\vec{l}, &
  \vec{A} &= \frac{\mu_0}{4\pi} \int_V \frac{\vec{J} dv}{R}
\end{align*}
With differential operators:
\begin{align*}
  \vec{E} &= - \grad \varphi, &
  \mu_0 \vec{J} &= - \vlaplacian \vec{A}.
\end{align*}
By taking the divergence on both sides of the equation with the electric field
we get \(\rho/\epsilon = - \laplacian \varphi\), which also contains the
Laplacian operator. We will study equations with of form in \S \ref{sec:poisson}.

% \subsection{Energy density}

\section{Laplace and Poisson's equations} \label{sec:poisson}

The so called \emph{Poisson's equation} has the form
\[
  \laplacian \varphi = - \frac{\rho}{\epsilon}.
\]
When the right side of the equation is zero, it is also known as \emph{Laplace's
equation}.

\subsection{Easy solutions of Laplace and Poisson's equations}

\subsubsection{Geometry with zenithal and azimuthal symmetries (\"Ubung 2)}

Suppose we have a geometry where, using spherical coordinates, there is a
symmetry such that the solution does not depend on \(\phi\) or \(\theta\).
Then Laplace's equation reduces down to
\[
  \laplacian \varphi = \frac{1}{r^2} \partial_r ( r^2 \partial_r \varphi) = 0,
\]
which has solutions of the form
\[
  \varphi(r) = \frac{C_1}{r} + C_2.
\]

\subsection{Geometry with azimuthal and translational symmetry (\"Ubung 3)}

Suppose that when using cylindrical coordinates, the solution does not depend
on \(\phi\) or \(z\). Then Laplace's equation becomes
\[
  \laplacian A_z = \frac{1}{r} \partial_r (r \partial_r A_z) = 0.
\]

\end{document}
